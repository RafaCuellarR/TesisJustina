El presente capítulo tiene por finalidad ofrecer una introducción general al trabajo de tesis, situando el contexto en el que se inscribe y los principios que sustentan su enfoque. Se expone, en primera instancia, el problema que motiva la investigación; se fundamenta su relevancia y pertinencia; y se delimitan los alcances del estudio.

En las últimas décadas, la robótica de servicio se ha consolidado como un área de investigación y desarrollo en constante expansión. La competencia internacional RoboCup@HOME se fundó con el objetivo de evaluar anualmente las capacidades de la tecnología emergente en robótica de servicio, incluyendo, entre otras, la interacción y cooperación humano-máquina, la navegación en ambientes dinámicos y la manipulación de objetos~\cite{robocup_athome}.

\section{Planteamiento del problema}

El Laboratorio de Bio-Robótica de la UNAM, fundado en 1996, emprendió en 2005 la construcción de robots móviles destinados a ayudar a las personas en sus actividades cotidianas, en entornos domésticos, académicos, laborales y clínicos. El robot de servicio de propósito general \emph{Justina} es resultado de la evolución de otros robots creados en el laboratorio~\cite{justina_uv}.

En la actualidad, \emph{Justina} cuenta con una cámara RGBD, un micrófono direccional, un torso telescópico, un sensor LiDAR, dos brazos manipuladores con siete grados de libertad cada uno y una base móvil FESTO \emph{Robotino 3}. No obstante, los manipuladores de \emph{Justina} exhiben deficiencias críticas que impidieron su participación en la RoboCup@HOME 2025. Los actuadores carecen de la capacidad de carga necesaria para las tareas de manipulación doméstica, los componentes estructurales presentan deformaciones permanentes por sobrecarga y el deterioro acumulado del sistema compromete su desempeño. Asimismo, aunque posee dos brazos, la manipulación colaborativa constituye un desafío técnico pendiente.

En respuesta a esta problemática, el presente trabajo plantea el rediseño e implementación integral de los subsistemas mecánico, electrónico y de software que conforman los brazos de \emph{Justina}, con el propósito de superar las limitaciones actuales y habilitar la manipulación colaborativa.

\section{Justificaci\'on}

El rediseño de los manipuladores de \emph{Justina} responde a múltiples necesidades del ámbito académico y del desarrollo tecnológico en robótica de servicio.

\emph{Justina} es un robot de servicio diseñado para operar en entornos domésticos, concebido como parte de una iniciativa para acercar la robótica a las personas y mejorar su calidad de vida. Sin embargo, la meta principal del proyecto es fomentar la formación científica e ingenieril de los estudiantes, brindándoles la oportunidad de aplicar sus conocimientos en escenarios prácticos utilizando tecnología de vanguardia\cite{BioRoboticsUNAM}. Los manipuladores son esenciales para estos fines, ya que constituyen la interfaz de interacción física con objetos y personas.  

La integración de manipulación colaborativa amplía significativamente el rango de tareas que \emph{Justina} puede realizar, permitiendo manejar objetos de mayor tamaño, peso y complejidad, y coordinar actividades que requieren ambos brazos, como la apertura de contenedores o el ensamblaje de objetos.

La participación en la RoboCup@HOME promueve el avance tecnológico del laboratorio al establecer estándares internacionales de desempeño y facilitar la comparación con otras instituciones académicas. No poder participar representa una pérdida de oportunidades de evaluación, retroalimentación y visibilidad.

Finalmente, este trabajo fortalece las competencias del Laboratorio de Bio-Robótica de la UNAM. Introduce la manipulación colaborativa como línea de investigación y desarrollo, genera conocimiento técnico documentado sobre el diseño mecatrónico de manipuladores y proporciona un marco para futuros avances.

\section{Alcance}

El presente trabajo abarca el rediseño integral de los brazos manipuladores de \emph{Justina} en tres disciplinas:

\begin{itemize}
    \item \textbf{Mecánica:} diseño de manipuladores y efector finales, análisis por elemento finito y manufactura.
    \item \textbf{Electrónica:} arquitectura del sistema de control, integración de sensores y administración de energía.
    \item \textbf{Software:} desarrollo de nodos de control de los manipuladores y efectores finales e infraestructura para la manipulación colaborativa.
\end{itemize}

Se contempla además la validación experimental mediante pruebas de desempeño en tareas representativas de robótica de servicio.

El alcance del proyecto no incluye modificaciones a otros subsistemas de \emph{Justina}. En cuanto a la manipulación colaborativa, se implementará la infraestructura y funcionalidad básica de coordinación, pero no se desarrollarán algoritmos avanzados de planificación, los cuales constituyen trabajo futuro.

\section{Objetivos}

Diseñar e implementar los subsistemas mecánico y electr\'onico de los brazos de un robot de servicio de prop\'osito general con el fin de manipular objetos de uso com\'un en un entorno dom\'estico, así como las interfaces para su integraci\'on en el \emph{stack} de software.

\subsection{Objetivos espec\'ificos}

\begin{itemize}
    \item Diseñar y manufacturar dos manipuladores rob\'oticos.
    \item Diseñar y manufacturar dos efectores finales (EOAT) para manipular objetos comunes de uso doméstico.
    \item Implementar el nodo de control en ROS para resolver el modelo de la cinem\'atica directa e inversa de la posición de los manipuladores.
    \item Implementar un nodo de control en ROS para tomar los objetos.
    \item Actualizar el paquete de ROS con la descripci\'on del robot.
\end{itemize}

\section{Organización del trabajo}