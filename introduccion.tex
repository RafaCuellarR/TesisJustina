El presente capítulo tiene por finalidad ofrecer una introducción general al trabajo de tesis, situando el contexto en el que se inscribe y los fundamentos que orientan su desarrollo. Se expone, en primera instancia, el problema que motiva la investigación; se fundamenta su relevancia y pertinencia; y se delimitan los alcances y limitaciones del estudio.

\section{Definición del problema}

En las últimas décadas, la robótica de servicio se ha consolidado como un área de investigación y desarrollo en constante expansión. La competencia internacional RoboCup@HOME se fundó con el objetivo de evaluar anualmente las capacidades de la tecnología emergente en robótica de servicio, incluyendo, entre otras, la interacción y cooperación humano-máquina, la navegación en ambientes dinámicos y la manipulación de objetos~\cite{robocup_athome}. 

El Laboratorio de Bio-Robótica de la UNAM, fundado en 1996, emprendió en 2005 el desarrollo de robots móviles destinados a ayudar a las personas en sus actividades cotidianas, en entornos domésticos, académicos, laborales y clínicos. El robot de servicio de propósito general \emph{Justina} es resultado de la evolución de otros robots creados en el laboratorio~\cite{justina_uv}.

En la actualidad, \emph{Justina} cuenta con una cámara RGBD, un micrófono direccional, un torso telescópico, un sensor LiDAR, dos brazos manipuladores con siete grados de libertad cada uno y una base móvil FESTO \emph{Robotino 3}. No obstante, los brazos exhiben deficiencias críticas que impidieron la participación de \emph{Justina} en la RoboCup@HOME 2025. Los actuadores carecen de la capacidad de carga necesaria para las tareas de manipulación doméstica, los componentes estructurales presentan deformaciones permanentes por sobrecarga y el deterioro acumulado compromete el desempeño del robot. Asimismo, la capacidad de manipulación colaborativa entre ambos brazos aún no ha sido implementada, constituyendo un desafío técnico pendiente.

En respuesta a esta problemática, el presente trabajo plantea el diseño y la implementación de mejoras integrales en los subsistemas mecánico, electrónico y de software que conforman los brazos manipuladores, con el propósito de fortalecer sus capacidades de manipulación de objetos y habilitar la capacidad de manipulación colaborativa.

\section{Justificaci\'on}

El diseño de los manipuladores de Justina responde a múltiples necesidades tanto del ámbito académico como del desarrollo tecnológico en robótica de servicio.

En primer lugar, la participación competitiva en RoboCup@HOME constituye un motor fundamental para el avance tecnológico del laboratorio, ya que establece estándares internacionales de desempeño y permite la comparación directa con desarrollos de instituciones de investigación a nivel mundial. La imposibilidad de participar en ediciones recientes debido a fallas en los manipuladores representa una pérdida de oportunidades de evaluación, retroalimentación y visibilidad para el proyecto Justina.

Desde una perspectiva técnica, los manipuladores constituyen uno de los subsistemas más críticos en robots de servicio, ya que determinan directamente las capacidades de interacción física con objetos y el entorno. Un diseño robusto y confiable es fundamental para ejecutar tareas de manipulación en escenarios domésticos, tales como asistencia en actividades de la vida diaria, apoyo en tareas de limpieza y organización, o colaboración en entornos de cuidado de personas.

Adicionalmente, el desarrollo de capacidades de manipulación coolaborativa amplía significativamente el rango de tareas que Justina puede realizar, permitiendo la manipulación de objetos de mayor tamaño, peso o complejidad, así como la ejecución de operaciones que requieren coordinación simultánea de ambos brazos, como apertura de contenedores, ensamblaje de objetos o transporte colaborativo.

Finalmente, este trabajo contribuye a la consolidación de una línea de investigación nacional en robótica de servicio, generando conocimiento técnico documentado sobre diseño mecatrónico de manipuladores y fortaleciendo las capacidades del Laboratorio de Bio-Robótica de la UNAM para futuros desarrollos en el área.

\section{Alcance}

Hasta dónde va a llegar el trabajo.