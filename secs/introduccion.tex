En 1920, el término \textbf{robot}, derivado de la palabra eslava \emph{robota}, que significa trabajo subordinado, fue introducido por el dramaturgo checo Karel Čapek en su obra \emph{Rossum's Universal Robot}~\cite{robotics_handbook}. Sin embargo, es hasta 1961 cuando se inicia la robótica como disciplina con la invención de los brazos autónomos para asistir a la manufactura en líneas de producción con el robot \emph{Unimate}~\cite{yazmin_intro}. Sus tareas se centraban en la manipulación de piezas y ensambles desde una posición fija en un entorno controlado, prácticamente estático, con la intención de relevar a los trabajadores de tareas monótonas e incrementar la producción.

Debido a esta reciente introducción e investigación en robótica, aún existen diversas definiciones de \textbf{robot}, que varían de acuerdo con el campo de aplicación y el área de investigación. Derivado del enfoque industrial inicial, las definiciones tanto de la \emph{International Standard Organization} en el estándar ISO 8373 como de la \emph{Robot Institute of America} reducen el término de robot a un manipulador\footnote{El estándar ISO 8373 define robot como: ``Un manipulador automáticamente controlado, reprogramable, multiuso, programable en tres o más ejes, que pueden estar fijos en un lugar o movilizarse para ser usado en aplicaciones de automatización industrial''; mientras que la  RIA considera a un robot como ``un manipulador multifuncional reprogramable diseñado para mover materiales, partes, herramientas o dispositivos especializados a través de diversos movimientos programados para la realización de diversas tareas''.}~\cite{yazmin_robotsServicio}. Por otro lado, Bajd~\cite{bajd_introRobotics} amplía la definición, considerando la robótica contemporánea, como la rama de la ciencia que estudia a aquellos sistemas cuya principal característica es el movimiento.

Durante las últimas décadas se han diseñado y construido una gran cantidad de robots en dos ejes principales: su grado de autonomía, entendida como la habilidad de tomar decisiones basadas en una representación interna del mundo, y su capacidad de actuar en entornos complejos. Ambos relacionados directamente con su inteligencia~\cite{yazmin_intro}.

\hypertarget{arkin}{De esta forma, la definición que resulta más útil para el desarrollo de este trabajo es la de Arkin~\cite{arkin}, quien propone que un robot inteligente es una máquina capaz de extraer información de su ambiente y usar el conocimiento acerca de su mundo para moverse de manera segura y significativa, con un propósito específico.}

\section{Robótica de servicio doméstico}

Los avances exponenciales en robótica han abierto la posibilidad de introducir robots en nuevos entornos. El objetivo de la robótica de servicio es reducir o eliminar el trabajo físico de las personas en lugares como casas. Los dispositivos robóticos tienen el potencial de ayudar a las personas con discapacidad a moverse o brindar compañía a ancianos. Adicionalmente, los robots pueden hacer más seguros los espacios con acciones que van desde recoger objetos para evitar tropiezos hasta verificar que se haya cerrado totalmente la llave de gas~\cite{yazmin_robotsServicio}.

La  definición de esta rama más reciente complementa la de \hyperlink{arkin}{Arkin} como una máquina móvil reprogramable autónoma o semiautónoma diseñada para operar en entornos dinámicos de manera segura, confiable y robusta, y ser capaz de realizar tareas específicas~\cite{yazmin_robotsServicio}.

Bill Gates~\cite{gates_article} vaticinó que en un hogar promedio habría robots de servicio doméstico con propósito específico encargados de cortar el césped, vigilar, limpiar, aspirar, trapear, lavar y planchar, como ya los hay en la actualidad. Pero también habría \textbf{robots humanoides de propósito general} que ayudarían a las personas a traer y llevar objetos. Este último tipo de robot tendría la capacidad de comunicarse con los miembros de la casa utilizando lenguaje natural y planear las acciones y movimientos necesarios con base en sus núcleos de acción predefinidos~\cite{yazmin_robotsServicio} para realizar sus funciones.

\hypertarget{basic_skills}{De esta forma, los robots de servicio doméstico deben contar con tres habilidades básicas: una navegación robusta, manipulación móvil y comunicación intuitiva con el usuario~\cite{nimbro_manipulation}}.

En 1997, a partir de avances significativos en inteligencia artificial y robótica, tales como la derrota del campeón mundial de ajedrez Gary Kaspárov a manos del \emph{IBM Deep Blue} y el despliegue del primer sistema robótico autónomo en Marte \emph{Sojourner}, surgió la RoboCup (Robot World Cup), una de las iniciativas con mayor relevancia a nivel internacional en el desarrollo de estos campos. Con el objetivo de definir metas a largo plazo y estandarizar plataformas de desarrollo, se coordinan anualmente las principales instituciones enfocadas en investigación alrededor del mundo para presentar sus avances~\cite{robocup_history}.

Debido al éxito de este proyecto, el alcance de la RoboCup se amplió con el surgimiento de diferentes ligas. RoboCup@HOME se fundó con el objetivo de evaluar las capacidades de la tecnología emergente aplicada a la mejora de las \hyperlink{basic_skills}{habilidades básicas en robótica de servicio}~\cite{robocup_athome}.

\section{Justificación}

La manipulación es un problema bien conocido en robótica. Constituye la interfaz de interacción física con objetos y personas, siendo uno de los medios más importantes de un robot para efectuar cambios en su entorno. Las tres principales funciones de una mano humana son explorar, sujetar y manipular objetos. La manipulación robótica se centra en resolver los últimos dos, ya que el primero cae en el terreno de la háptica~\cite{bicchi_grasping}.

La interacción física presente en nuestra vida diaria cuando se realiza una manipulación, incluso en tareas simples y comunes, va más allá de tomar y colocar algún objeto. Ejemplos de estas tareas son encender la luz, extraer un libro de una estantería, abrir una puerta o cajón, empujar algún  objeto, entre otros~\cite{jaume2_manipulation}.

\emph{Justina} es un robot de servicio diseñado en el Laboratorio de Bio-Robótica de la UNAM para operar en entornos domésticos, concebido como parte de una iniciativa para acercar la robótica a las personas y mejorar su calidad de vida~\cite{BioRoboticsUNAM}. Para resolver la manipulación cuenta con dos brazos de siete grados de libertad cada uno, lo que le permitiría transportar una mayor cantidad de objetos en un menor tiempo y llevar a cabo tareas más complejas.

Contar con dos brazos completamente funcionales abre la puerta a la integración de la manipulación colaborativa. Esto permitiría al robot manejar objetos de mayor tamaño, peso y complejidad, e incorporar acciones como la apertura de contenedores o la manipulación fina de objetos.

\section{Planteamiento del problema}

Los manipuladores de \emph{Justina} exhiben deficiencias críticas que impidieron su participación en la RoboCup@HOME 2025. Los actuadores carecen de la capacidad de carga necesaria para las tareas de manipulación doméstica, los componentes estructurales presentan deformaciones permanentes por sobrecarga y el deterioro acumulado del sistema compromete su desempeño. Asimismo, aunque posee dos brazos, la manipulación colaborativa constituye un desafío técnico pendiente.

En respuesta a esta problemática, el presente trabajo plantea el rediseño e implementación integral de los subsistemas mecánico, electrónico y de software que conforman los brazos de \emph{Justina}.

\section{Alcance}

El presente trabajo abarca el rediseño integral de los brazos manipuladores de \emph{Justina} en tres disciplinas:

\begin{itemize}
    \item \textbf{Mecánica:} diseño de manipuladores y efector finales, análisis por elemento finito y manufactura.
    \item \textbf{Electrónica:} arquitectura del sistema de control, integración de sensores y administración de energía.
    \item \textbf{Software:} desarrollo de nodos de control de los manipuladores y efectores finales e infraestructura para la manipulación colaborativa.
\end{itemize}

Se contempla además la validación experimental mediante pruebas de desempeño en tareas de manipulación simple en robótica de servicio.

\section{Objetivos}

Diseñar e implementar los subsistemas mecánico y electrónico de los brazos de un robot de servicio de propósito general con el fin de manipular objetos de uso común en un entorno doméstico, así como las interfaces para su integración en el \emph{stack} de software.

\subsection{Objetivos específicos}

\begin{itemize}
    \item Diseñar y manufacturar dos manipuladores robóticos.
    \item Diseñar y manufacturar dos efectores finales (EOAT) para manipular objetos comunes de uso doméstico.
    \item Implementar el nodo de control en ROS para resolver el modelo de la cinemática directa e inversa de la posición de los manipuladores.
    \item Implementar un nodo de control en ROS para tomar los objetos.
    \item Actualizar el paquete de ROS con la descripción del robot.
\end{itemize}

\section{Organización del trabajo}

A continuación se presenta una visión general de la estructura del presente trabajo, ofreciendo una breve descripción del contenido y propósito de cada uno de los capítulos que lo conforman.

En el Capítulo 2 se revisan los fundamentos teóricos y técnicos que sirven de base para el desarrollo del proyecto. Se abordan los temas de mecanismos y manipuladores robóticos, control de posición y trayectoria, medición de par y fuerza en las juntas, así como el uso del \emph{Robot Operating System} (ROS) como plataforma de desarrollo.

En el Capítulo 3 se describe el proceso metodológico seguido para el diseño del sistema. Incluye el levantamiento de necesidades, la generación de especificaciones técnicas, el análisis comparativo de soluciones existentes mediante \emph{benchmarking}, y el desarrollo del diseño conceptual. Asimismo, se detallan las herramientas utilizadas, como la matriz \emph{Quality Function Deployment} (QFD), para la selección y evaluación de alternativas de diseño.

En el Capítulo 4 se presenta el desarrollo integral del sistema desde una perspectiva mecatrónica. El diseño mecánico comprende la estructura del manipulador y el efector final, además de los análisis por elemento finito y el proceso de manufactura. El diseño eléctrico abarca la arquitectura del sistema de control y la administración de energía. Finalmente, se detalla la instrumentación del software y la actualización de los nodos de control del robot \emph{Justina}.

En el Capítulo 5 se describe el protocolo de validación del sistema mecánico y electrónico, así como las pruebas de manipulación implementadas para evaluar el desempeño del robot y sus resultados.

En el Capítulo 6 se presentan las conclusiones derivadas del trabajo, acompañadas de un análisis de los resultados y una discusión sobre los logros alcanzados. También se destacan las principales contribuciones del proyecto y se proponen posibles líneas de trabajo futuro que podrían continuar y mejorar el desarrollo realizado.