\section{Robótica}

La humanidad ha tenido, a lo largo de su historia, la inquietud de crear máquinas a su imagen y semejanza con habilidad e inteligencia. Esto queda plasmado en mitos como la leyenda del titán Prometeo, quien moldeó a la humanidad con arcilla, o la del gigante Talus, un esclavo de bronce hecho por Hefesto. También, hay avances tecnológicos que abonaron a este fin como el clepsidra, un reloj automático que mide el tiempo basado en el flujo de un fluido; el teatro de Herón de Alejandría, que recreaba obras enteras sin intervención humana usando autómatas; y los ingeniosos diseños y prototipos de Leonardo Da Vinci.~\cite{handbook_robotics}

\subsection{Definición}

Tadej Bajd, en su libro \emph{Introduction to robotics}~\cite{bajd_intro}, define a la robótica contemporánea como la rama de la ciencia que estudia a aquellos sistemas cuya principal característica es el movimiento. En su libro del mismo nombre, John Craig se centra en definir al \textbf{robot industrial} como extensión de las nuevas técnicas de automatización que marcaron las rápidas transformaciones y acelerado crecimiento económico del siglo XX~\cite{craig_intro}. Lo que constituye a un robot industrial está en debate y el criterio en el que normalmente se basa la clasificación entre la máquina y robot industrial

\subsection{Clasificación}


\subsection{RoboCup@HOME}